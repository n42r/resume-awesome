%!TEX TS-program = xelatex
%!TEX encoding = UTF-8 Unicode
% Awesome CV LaTeX Template for CV/Resume
%
% This template has been downloaded from:
% https://github.com/posquit0/Awesome-CV
%
% Author:
% Claud D. Park <posquit0.bj@gmail.com>
% http://www.posquit0.com
%
% Template license:
% CC BY-SA 4.0 (https://creativecommons.org/licenses/by-sa/4.0/)
%


%-------------------------------------------------------------------------------
% CONFIGURATIONS
%-------------------------------------------------------------------------------
% A4 paper size by default, use 'letterpaper' for US letter
\documentclass[12pt, a4paper]{awesome-cv}

% Configure page margins with geometry
\geometry{left=1.4cm, top=.8cm, right=1.4cm, bottom=1.8cm, footskip=.5cm}

% Color for highlights
% Awesome Colors: awesome-emerald, awesome-skyblue, awesome-red, awesome-pink, awesome-orange
%                 awesome-nephritis, awesome-concrete, awesome-darknight
\colorlet{awesome}{awesome-darknight}
% Uncomment if you would like to specify your own color
%\definecolor{awesome}{HTML}{f23a49}

% Colors for text
% Uncomment if you would like to specify your own color
% \definecolor{darktext}{HTML}{414141}
% \definecolor{text}{HTML}{333333}
% \definecolor{graytext}{HTML}{5D5D5D}
% \definecolor{lighttext}{HTML}{999999}
% \definecolor{sectiondivider}{HTML}{5D5D5D}

% Set false if you don't want to highlight section with awesome color
\setbool{acvSectionColorHighlight}{true}

% If you would like to change the social information separator from a pipe (|) to something else
\renewcommand{\acvHeaderSocialSep}{\quad\textbar\quad}




%-------------------------------------------------------------------------------
% USER DEFINED (NK)


\definecolor{offwhite}{RGB}{251,247,245}

\pagecolor{offwhite}

\definecolor{lblue}{RGB}{59, 83, 239} 
%\definecolor{lblue}{RGB}{58, 73, 242} 


\newcommand{\see}[1]{\textit{(See \urll{#1} for more)}}
\newcommand{\urll}[1]{\textcolor{lblue}{\underline{\href{https://#1}{#1}}}}


% END OF USER DEFINED
%-------------------------------------------------------------------------------






%-------------------------------------------------------------------------------
%	PERSONAL INFORMATION
%	Comment any of the lines below if they are not required
%-------------------------------------------------------------------------------
% Available options: circle|rectangle,edge/noedge,left/right
\photo[rectangle,edge,right]{./pic-600-box.jpg}
\name{Nasr Kasrin (PhD)}{}
\position{Software Architect{\enskip\cdotp\enskip}Product Strategist}


\address{96050 Bamberg, Germany}


%\born{January 1986}

\mobile{(+49) 176-3664-2113}                  
\email{n42r.me@gmail.com}
%\social[linkedin]{nkasrin}                         % optional, remove / comment the line if not wanted
%\social[github]{n42r}                              % optional, remove / comment the line if not wanted

%\phone[fixed]{+2~(345)~678~901}
%\phone[fax]{+3~(456)~789~012}

\homepage{n42r.github.io}



%\mobile{(+82) 10-9030-1843}
%\email{posquit0.bj@gmail.com}
%\dateofbirth{January 1st, 1970}
%\homepage{www.posquit0.com}
%\github{posquit0}
%\linkedin{posquit0}
% \gitlab{gitlab-id}
% \stackoverflow{SO-id}{SO-name}
% \twitter{@twit}
% \skype{skype-id}
% \reddit{reddit-id}
% \medium{madium-id}
% \kaggle{kaggle-id}
% \hackerrank{hackerrank-id}
% \googlescholar{googlescholar-id}{name-to-display}
%% \firstname and \lastname will be used
% \googlescholar{googlescholar-id}{}
% \extrainfo{extra information}

%\quote{``Be the change that you want to see in the world."}


%-------------------------------------------------------------------------------
\begin{document}

% Print the header with above personal information
% Give optional argument to change alignment(C: center, L: left, R: right)
\makecvheader[C]

% Print the footer with 3 arguments(<left>, <center>, <right>)
% Leave any of these blank if they are not needed
%\makecvfooter
%  {\today}
%  {Byungjin Park~~~·~~~Résumé}
%  {\thepage}


%-------------------------------------------------------------------------------
%	CV/RESUME CONTENT
%	Each section is imported separately, open each file in turn to modify content
%-------------------------------------------------------------------------------
%-------------------------------------------------------------------------------
%	SECTION TITLE
%-------------------------------------------------------------------------------

%\cvsection{Summary}
%\vspace{1ex}


%-------------------------------------------------------------------------------
%	CONTENT
%-------------------------------------------------------------------------------
\begin{cvparagraph}

%---------------------------------------------------------


%Dynamic architect and technology leader with a history of guiding successful projects spanning diverse domains (data architecture, social networks, mobile apps, games, robotics, and AI), team sizes (2 to 10), and settings (from agile B2C environments to R\&D projects backed by millions of Euros in public funding). With expertise at the intersection of engineering and product leadership, I excel in delivering cutting-edge solutions that drive business growth and enhance user experiences \see{n42r.github.io}.

%\textbf{Strengths}: Thinking (Strategic, Critical, Creative), Emotion (Enthusiastic, Resilient), Execution (Initiative, Decisive), Relation (Empathy, Collaborate)


%Dynamic architect and technology lead with a history of guiding successful projects across diverse domains (data architecture, AI, robotics, mobile apps, and games) and settings (from agile B2C environments to multi-million R\&D projects). I bring a strategic and creative approach to problem-solving, coupled with critical thinking that ensures robust and innovative solutions that drive business growth and enhance user experiences \see{n42r.github.io}.


%Dynamic architect, research scientist, and innovative technology lead with a history of guiding successful projects across diverse domains (data architecture, AI, robotics, mobile apps, and games) and settings (from agile B2C environments to multi-million R\&D projects). I bring a strategic and creative approach to problem-solving, coupled with decisive initiative and collaborative empathy which fosters success \see{n42r.github.io}.
%I excel in delivering cutting-edge solutions that drive business growth and enhance user experiences 

%With over 15 years of experience in technology innovation, I have led successful initiatives in AI, data architecture, e-commerce, robotics, and more across startups, global corporations, and large-scale R\&D projects. Anchored by advanced academic credentials--including a recent PhD in data architecture and management--I blend hands‑on expertise with a leadership style centered on exploration, experimentation, and strategic risk‑taking. This diverse background enables me to drive business growth and deliver state‑of‑the‑art solutions that elevate user experiences.


\end{cvparagraph}

\vspace{-0.5ex}

%-------------------------------------------------------------------------------
%	SECTION TITLE
%-------------------------------------------------------------------------------
\cvsection{Projects}


%-------------------------------------------------------------------------------
%	CONTENT
%-------------------------------------------------------------------------------
\begin{cventries}


  \cventry
    {Open Source Software} % Project Type
    {Muze AI} % Project Name
    {2023 - 2024} % URL
    {\urll{github.com/n42r/muze-ai}} % Date(s)
    {
          \begin{cvitems} % Description(s) of tasks/responsibilities
	        \item {Developed an early AI/LLM music discovery tool, bypassing the reliance on Spotify's music recommendation metadata.}
	      \end{cvitems}
    }

%---------------------------------------------------------


  \cventry
    {Open Source Software} % Project Type
    {Guestrrday} % Project Name
    {2022 - 2023} % URL
    {\urll{github.com/n42r/guestrrday}} % Date(s)
    {
          \begin{cvitems} % Description(s) of tasks/responsibilities
	        \item {Engineered an open-source music tagging tool capable of processing over 20,000 items, enhancing music organization and accessibility.}
	      \end{cvitems}
    }


%---------------------------------------------------------


  \cventry
    {Architectural Model} % Project Type
    {The Basin Network (Doctoral Project)} % Project Name
    {2019 - 2023} % URL
    {\urll{n42r.github.io/phd}} % Date(s)
    {
          \begin{cvitems} % Description(s) of tasks/responsibilities
			\item {Invented a novel architectural pattern for data cataloging, improving on data mesh and data space models in 3 respects: precision, level of detail, and generalization, culminating in the award of a PhD with distinction and 2 published papers.}
%			\item {Standardized the pattern to facilitate data exchange, data governance, interoperability, and effective management of data as an asset, .}
			%\item {Published findings in 2 journals, contributing to advancements in the field of data management.}
	      \end{cvitems}
    }


%---------------------------------------------------------

  \cventry
    {SaaS (European Union Publicly Funded Project)} % Project Type
    {SIMUTOOL Data Lake} % Project Name
    {2015 - 2019} % URL
    {\urll{github.com/simutool}} % Date(s)
    {
          \begin{cvitems} % Description(s) of tasks/responsibilities
			\item {Built a data lake SaaS for a consortium of 8 manufacturing companies, cutting turnover time by 30\% by fostering data-driven cooperation.} 
%			\item {Developed the back-end server (python, Flask, WSGI, REST, Neo4J, Semantic Data Modeling, RDF, JSON-LD) as well as a CRUD web application (Apache Server, web2py framework, HTML, CSS, MySQL).}
			%\item {Enabled streamlined data sharing \& exchange in a consortium of 8 companies}
			%\item {Designed the application architecture, developed the back-end, and supervised a team in creating client applications, contributing significantly to the project's success, as recognized by an expert panel appointed by the European Commission.}
%			\item{Designed the application, constructed the back-end, and guided client development, pivotal to the project's success as acknowledged by an EC-appointed panel.}
	%		\item {Designed the application, constructed the back-end, and guided client development, pivotal to the project's success as per an EC-appointed panel.}
			\item {Designed the app, built the back-end, and led client development, crucial to project success as acknowledged by an EC-appointed panel.}
	      \end{cvitems}
%	      90\% user satisfaction 
    }
    

%---------------------------------------------------------


  \cventry
    {Social Network/iOS Mobile App (B2C)} % Project Type
    {Greetings Studio} % Project Name
    {2012 - 2014} % URL
    {\urll{n42r.github.io/gs}} % Date(s)
    {
          \begin{cvitems} % Description(s) of tasks/responsibilities
				\item {Directed the development of an E-greeting card social network and mobile app, optimizing user engagement leading to 10-fold growth in users.}
				\item {Redesigned the architecture, leading to a 15\% reduction in development time originally due to circumventing technical architecture debt.}
	      \end{cvitems}
    }
    
    
%---------------------------------------------------------


  \cventry
    {Game/iOS Mobile App (B2C)} % Project Type
    {Tawla (Backgammon Board Game)} % Project Name
    {2011 - 2013} % URL
    {\urll{n42r.github.io/tw}} % Date(s)
    {
          \begin{cvitems} % Description(s) of tasks/responsibilities
	          	\item {Elevated app ratings from 3 to 4.5/5 stars by the implementation of advanced AI players, enhancing gameplay dynamics.}
				\item {Achieved a 10\% increase in positive reviews due to building a sophisticated random number generator for the die, which improved gameplay.}
	      \end{cvitems}
    }
    
    

%---------------------------------------------------------


  \cventry
    {AI/Robotics software to compete in the 2011 International RoboCup Competition} % Project Type
    {ArtSapiens 2D Soccer} % Project Name
    {2010 - 2011} % URL
    {\urll{ssim.robocup.org}} % Date(s)
    {
          \begin{cvitems} % Description(s) of tasks/responsibilities
			\item {Co-founded and co-led a team of 10 that developed robotics/AI software which qualified to compete in the 2011 RoboCup Competition.}%, which qualifications and earned a spot in 2011 competition in Kuala Lumpur, Malaysia.}
	      \end{cvitems}
    }
    
    
%---------------------------------------------------------

\end{cventries}

%-------------------------------------------------------------------------------
%	SECTION TITLE
%-------------------------------------------------------------------------------
%\clearpage 
\cvsection{Work Experience}


%-------------------------------------------------------------------------------
%	CONTENT
%-------------------------------------------------------------------------------
\begin{cventries}

%---------------------------------------------------------


  \cventry
    {University of Bamberg (Third-party Funded Project)} % Job title
    {Research Associate (Architect | Team Lead)} % Organization
    {2015 - 2020} % Location
    {Bamberg, Germany} % Date(s)
    {
      \begin{cvitems} % Description(s) of tasks/responsibilities
		\item {Led a 4-year project to develop a data management SaaS for a €3.5 million 8-company EU manufacturing project, optimizing data discovery, collaboration, and turnaround time, and resulting in enhanced operational efficiency \see{github.com/simutool}.}
		\item {Cultivated close relationships with 10+ external partners, facilitating deep domain understanding and precise requirements identification.}
		\item {Engineered a read-heavy, horizontally scalable SaaS, ensuring seamless operations and future-proof architecture (ex., stateless nodes).}
      \end{cvitems}
    }

%---------------------------------------------------------


%  \cventry
%    {University of Technology Twintech} % Job title
%    {Assistant Dean / Lecturer} % Organization
%    {2013 - 2015} % Location
%    {Sanaa, Yemen} % Date(s)
%    {
%      \begin{cvitems} % Description(s) of tasks/responsibilities
%		\item {Managed the IT and Multimedia faculties, reporting to the university president}
%		\item {Modernized the curriculum by leading a curriculum development initiative for the Business IT and Multimedia faculties}
%		\item {Taught several Courses: \emph{Software Analysis}, \emph{Human-Computer Interaction}, \emph{Information Design}, etc}
%      \end{cvitems}
%    }

%---------------------------------------------------------


  \cventry
    {TayaIT} % Job title
    {Team Leader | Software Architect | R\&D Engineer} % Organization
    {2011 - 2014} % Location
    {Cairo, Egypt} % Date(s)
    {
      \begin{cvitems} % Description(s) of tasks/responsibilities
		\item {Directed agile technical teams, ranging from 2 to 5 members, in the development of two enduring social media/mobile products, driving perpetual augmentation of app rankings (4.5 starts) and a 10-fold increase in user engagement (See 'Greetings Studio' and 'Tawla' in Projects).}
		\item {Reduced feedback-development cycle times by 25\% by coordinating cross-functional collaboration between technical, business, and UI/UX teams, streamlining workflows and fostering tighter cooperation and heightened productivity.}
		\item {Saved the company over 10-man months by investigating emerging technologies and alternative project paths and advising the CEO in adopting better paths or avoiding dead-ends and sub-optimum paths.}
      \end{cvitems}
    }

%---------------------------------------------------------


%\cventry{2008 -- 2012}{Teaching Assistant}{The German University in Cairo (GUC)}{Cairo}{Egypt}{
%\begin{itemize}
%\item Taught 10 hours per week of tutorials, interfacing with 150 new students per semester.
%\item Courses: \emph{Introduction to Artificial Intelligence}, \emph{Introduction to Computer Science}, \emph{Introduction to Computer Programming}. 
%\end{itemize}}


\end{cventries}

%-------------------------------------------------------------------------------
%	SECTION TITLE
%-------------------------------------------------------------------------------
\cvsection{Skills}


%-------------------------------------------------------------------------------
%	SUBSECTION TITLE
%-------------------------------------------------------------------------------
%\cvsubsection{Databases}


%-------------------------------------------------------------------------------
%	CONTENT
%-------------------------------------------------------------------------------
\begin{cvskills}


\cvskill{DEVELOPMENT}{Python, Flask, WSGI, Apache, Docker (multi-container), Linux CLI, git, and low-/no-code. Formerly, C++, Lisp, Java.}

\cvskill{DATABASES}{MongoDB, Neo4j, MySQL, Google Firebase, and Resource Description Framework (RDF) / Semantic Data.}

\cvskill{LEADERSHIP}{Empathetic leadership and coaching, agile/lean project leadership, change management.}

\cvskill{ARCHITECTURE}{Distributed architectures, modular monoliths (clean/hexagonal architecture), HTTP API interface design.}

\cvskill{DOMAIN EXPERTISE}{Data integration, governance, interoperability, standardization, cataloguing, domain modeling, robotics/AI.}

\cvskill{SOFT SKILLS}{Critical and conceptual thinking, collaboration/teamwork, clear communication of complex concepts.}

%\cvskill{OTHER}{}


\end{cvskills}

%-------------------------------------------------------------------------------
%	SECTION TITLE
%-------------------------------------------------------------------------------
\cvsection{Education}



\begin{cvskills}


\cvskill{PhD. (Dr. rer. nat.)}{Faculty of Information Systems \& Applied Computer Science, Otto-Friedrich-Universit\"{a}t Bamberg, Germany\hspace{15pt}\emph{2023}}

\cvskill{Masters of Science}{Faculty of Computer Science \& Engineering, German University in Cairo, Egypt\hspace{118pt}\emph{2010}}

\cvskill{Bachelors of Science}{Faculty of Computer Science \& Engineering, German University in Cairo, Egypt\hspace{118pt}\emph{2009}}

\end{cvskills}

\vspace{-0.5ex}


%-------------------------------------------------------------------------------
%	CONTENT
%-------------------------------------------------------------------------------
%\begin{cventries}

%---------------------------------------------------------
%  \cventry
%    {PhD. (Dr. rer. nat.), Faculty of Information Systems \& Applied Computer Science} % Degree
%    {Otto-Friedrich-Universit\"{a}t Bamberg} % Institution
%    {2020 - 2023} % Location
%    {Bamberg, Germany} % Date(s)
%    {
%    }

%---------------------------------------------------------
%  \cventry
%    {MSc. (Masters of Science), Faculty of Computer Science \& Engineering} % Degree
%    {The German University in Cairo} % Institution
%    {2010} % Location
%    {Cairo, Egypt} % Date(s)
%    {
%    }

%---------------------------------------------------------
%  \cventry
%    {BSc. (Bachelors of Science), Faculty of Computer Science \& Engineering} % Degree
%    {The German University in Cairo} % Institution
%    {2003 - 2008} % Location
%    {Cairo, Egypt} % Date(s)
%    {
%    }

%---------------------------------------------------------

%\end{cventries}

%%-------------------------------------------------------------------------------
%	SECTION TITLE
%-------------------------------------------------------------------------------
\cvsection{Publications}


%-------------------------------------------------------------------------------
%	SUBSECTION TITLE
%-------------------------------------------------------------------------------
%\cvsubsection{Databases}


%-------------------------------------------------------------------------------
%	CONTENT
%-------------------------------------------------------------------------------

%\begin{cvparagraph}

%Full list available at: \urll{scholar.google.com/citations?user=JgS4-1cAAAAJ}

%\end{cvparagraph}


\begin{cvskills}

\cvskill{2023}{The Basin Network: A Model for Data Sharing \& Exchange. \emph{PhD Thesis}}

\cvskill{2021}{Data-Sharing Markets for Integrating IoT Data Processing Functionalities. \emph{CCF Transactions on Pervasive Computing \&  Interaction}}

\cvskill{2018}{Semantic Data Management for Experimental Manufacturing Technologies. \emph{Datenbank-Spektrum}}

\cvskill{2010}{Focused Belief Revision as a Model of Fallible Relevance-Sensitive Perception. \emph{K\"{u}nstliche Intelligenz} (KI)}

\cvskill{2010}{High-Level Perception as Focused Belief Revision. \emph{European Conference on Artificial Intelligence (ECAI)}}

\end{cvskills}


\input{resume/languages.tex}


%-------------------------------------------------------------------------------
\end{document}
