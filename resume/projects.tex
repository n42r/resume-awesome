%-------------------------------------------------------------------------------
%	SECTION TITLE
%-------------------------------------------------------------------------------
\cvsection{Projects \scriptsize{\emph{(Titles are hyperlinks)}}}


%-------------------------------------------------------------------------------
%	CONTENT
%-------------------------------------------------------------------------------
\begin{cventries}

  \cventrythin
    {\urlll{github.com/n42r/guestrrday}{Guestrrday} \& \urlll{github.com/n42r/muze-ai}{Muze AI}} % Project Type
    {2022 - 2024} % Date(s)
    {
          \begin{cvitems} % Description(s) of tasks/responsibilities
	        \item {Developed personal open source projects: a music tagging tool (executed 20.000+ inputs) \& an early AI-powered music discovery prototype.}
	      \end{cvitems}
    }

%---------------------------------------------------------


%  \cventrythin
%    {\urlll{github.com/n42r/muze-ai}{Muze AI}} % Project Type
%    {2023 - 2024} % Date(s)
%    {
%          \begin{cvitems} % Description(s) of tasks/responsibilities
%	        \item {Developed an early AI/LLM music discovery tool, bypassing the reliance on Spotify's music recommendation metadata.}
%	      \end{cvitems}
%    }

%---------------------------------------------------------

%  \cventrythin
%    {\urlll{github.com/n42r/guestrrday}{Guestrrday}} % Project Type
%    {2022 - 2023} % Date(s)
%    {
%          \begin{cvitems} % Description(s) of tasks/responsibilities
%	        \item {Engineered an open-source music tagging tool to enhance music organization, which has successfully processed over 20,000 inputs %so far.}
%	      \end{cvitems}
%    }


%---------------------------------------------------------


  \cventrythin
    {\urlll{n42r.github.io/phd}{The Basin Network (Doctoral Project)}} % Project Type
    {2019 - 2023} % Date(s)
    {
          \begin{cvitems} % Description(s) of tasks/responsibilities
			\item {Invented a novel architectural pattern for data cataloging, improving on data mesh, data lake, and data space models in 3 respects: precision, level of detail, and generalization, culminating in the award of a PhD with distinction and 2 published papers.}
%			\item {Standardized the pattern to facilitate data exchange, data governance, interoperability, and effective management of data as an asset, .}
			%\item {Published findings in 2 journals, contributing to advancements in the field of data management.}
	      \end{cvitems}
    }


%---------------------------------------------------------

  \cventrythin
    {\urlll{github.com/simutool}{SIMUTOOL Data Lake}} % Project Type
    {2015 - 2019} % Date(s)
    {
          \begin{cvitems} % Description(s) of tasks/responsibilities
			\item {Led the development of data management SaaS for a consortium, cutting turnover time by 30\% by fostering data-driven cooperation.} 
%			\item {Developed the back-end server (python, Flask, WSGI, REST, Neo4J, Semantic Data Modeling, RDF, JSON-LD) as well as a CRUD web application (Apache Server, web2py framework, HTML, CSS, MySQL).}
			%\item {Enabled streamlined 	data sharing \& exchange in a consortium of 8 companies}
			%\item {Designed the application architecture, developed the back-end, and supervised a team in creating client applications, contributing significantly to the project's success, as recognized by an expert panel appointed by the European Commission.}
%			\item{Designed the application, constructed the back-end, and guided client development, pivotal to the project's success as acknowledged by an EC-appointed panel.}
	%		\item {Designed the application, constructed the back-end, and guided client development, pivotal to the project's success as per an EC-appointed panel.}
			\item {Designed the app, built the back-end, and led client development, crucial to project success as acknowledged by an EC-appointed panel.}
	      \end{cvitems}
%	      90\% user satisfaction 
    }
    

%---------------------------------------------------------


  \cventrythin
    {\urlll{n42r.github.io/gs}{Greetings Studio} \& \urlll{n42r.github.io/tw}{Tawla (Backgammon Board Game)}} % Project Type
    {2011 - 2014} % Date(s)
    {
          \begin{cvitems} % Description(s) of tasks/responsibilities
				\item {Directed the development of an E-greeting card social network and mobile app, optimizing user engagement leading to 10-fold growth in users.}
	          	\item {Elevated app ratings of a mobile game from 3 to 4.5/5 stars by the implementation of advanced AI players, enhancing gameplay dynamics.}
	      \end{cvitems}
    }


%---------------------------------------------------------


%  \cventrythin
%    {\urlll{n42r.github.io/gs}{Greetings Studio}} % Project Type
%    {2012 - 2014} % Date(s)
%    {
%          \begin{cvitems} % Description(s) of tasks/responsibilities
%				\item {Directed the development of an E-greeting card social network and mobile app, optimizing user engagement leading to 10-fold growth in users.}
%				\item {Redesigned the architecture, leading to a 15\% reduction in development time originally due to circumventing technical architecture debt.}
%	      \end{cvitems}
%    }
    
    
%---------------------------------------------------------


%  \cventrythin
%    {\urlll{n42r.github.io/tw}{Tawla (Backgammon Board Game)}} % Project Type
%    {2011 - 2013} % Date(s)
%    {
%          \begin{cvitems} % Description(s) of tasks/responsibilities
%	          	\item {Elevated app ratings from 3 to 4.5/5 stars by the implementation of advanced AI players, enhancing gameplay dynamics.}
%				\item {Achieved a 10\% increase in positive reviews due to building a sophisticated random number generator for the die, which improved gameplay.}
%	      \end{cvitems}
%    }
    
    

%---------------------------------------------------------


%  \cventrythin
%    {\urlll{ssim.robocup.org}{ArtSapiens 2D Soccer}} % Project Type
%    {2010 - 2011} % URL
%    {
%          \begin{cvitems} % Description(s) of tasks/responsibilities
%			\item {Co-founded \& co-led a team of 10 to develop robotics/AI software leading to qualification in the 2011 International RoboCup 2D League.}%, which qualifications and earned a spot in 2011 competition in Kuala Lumpur, Malaysia.}
%	      \end{cvitems}
%    }
    
    
%---------------------------------------------------------

\end{cventries}

\vspace{-0.5ex}
